\documentclass[notes.tex]{subfiles}
 
\begin{document}

\section{Domain and tree construction}
In the previous section we described the FleCSI framework.
This part gives details on the domain decomposition using octree, tree 
construction and search algorithm.

\subsection{Domain decomposition}
In the current version of FleCSI the domain decomposition is done using the
Morton ordering.
It allows to describe, sort and distribute particles based on a unique value
-- key.

In principle, various space-filling curves can be used, with their own
advantages and flaws:
\begin{itemize}
\item Morton ordering: interlace the bits of X, Y and Z positions to create
the key. It is very simple to compute, but the curve suffers discontinuous
jumps.
\item Hilbert-Peano: interlace the bits, but also add rotations. In this
ordering, data locality in memory guarantees locality in space.
\item Other space-filling curves: hexagonal space filling curves, ...?
\end{itemize}

This first implementation is based on the Morton ordering which is used during
several steps:
\begin{itemize}
\item The distribution part, to be able to split the particles between the
processes providing a good locality in the data.
\item The tree construction and search. 
\end{itemize}

\subsection{Binary, Quad- and Octrees}
% Describe the tree and the splitting version of it
% Explain the ghosts, exclusive and shared particles.


\end{document}
